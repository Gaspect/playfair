\documentclass{article}


\usepackage[utf8]{inputenc}
\usepackage[T1]{fontenc}
\usepackage[spanish,es-ucroman]{babel}
\usepackage[linktocpage,breaklinks,colorlinks,%
linkcolor=black,anchorcolor=black,citecolor=black,%
filecolor=black,menucolor=black,runcolor=black,%
urlcolor=black]{hyperref}
\usepackage{bookmark}

\title{Algoritmo de Playfair}
\author{Alejandro Valoy Gorrín, Jesús Enrique Fuentes González}

\begin{document}
    \maketitle
    El cifrado de Playfair fue el primer sistema de cifrado en encriptar
    pares de letras. Wheatstone inventó el cifrado para encriptar mensajes
    enviados por telegrama, pero lleva el nombre de su amigo lord Playfair,
    quien lo promovió para uso militar. 

    \section{Proceso de cifrado}
    \subsection{Creación de matriz de cifrado}
    Dada una llave \textit{k} a esta se le aplican las siguientes transformaciones
    \begin{enumerate}
        \item Convertir las letras de \textit{k} en mayúscula
        \item Eliminar los espacios en \textit{k}
        \item Eliminar las Ñ  en \textit{k}
        \item Reemplazar las J por I en \textit{k}
        \item Rellenar \textit{k} con el resto de letras del alfabeto(la longitud del alfabeto es de 25 para el playfair clásico) que no se repitan en \textit{k}.
        \item Convertir \textit{k} rellenada en una matriz \textit{m} de 5x5 donde 5 letras consecutivas representan una fila de la matriz \textit{m}. 
    \end{enumerate}
    \subsection{Cifrar texto}
    Dado un texto \textit{t} y una matriz de cifrado \textit{m}.
    \begin{enumerate}
        \item Convertir las letras de  \textit{t} en mayúsculas.
        \item Eliminar los espacios en \textit{t}.
        \item Eliminar las Ñ en \textit{t}.
        \item Reemplazar las J por I en \textit{t}.
        \item Adicionar la letra X a \textit{t} si la longitud de \textit{t} es impar.
        \item Dividir a \textit{t} en pares de caracteres consecutivos en \textit{m1} y  \textit{m2}.
        \item Por cada par de caracteres \textit{m1} y \textit{m2} hacer:
            \begin{enumerate}
                \item Si \textit{m1} y \textit{m2} se encuentran en la misma fila, escoger \textit{c1} y \textit{c2} situados a su derecha (circularmente).
                \item Si \textit{m1} y \textit{m2} se encuentran en la misma columna, escoger \textit{c1} y \textit{c2} situados debajo (circularmente).
                \item Si \textit{m1} y \textit{m2} se encuentran en distintas filas y columnas, escoger \textit{c1} y \textit{c2} situados en la diagonal opuesta (siempre de derecha a izquierda).
            \end{enumerate}
    \end{enumerate}
    \section{Debilidades criptográficas}
    \begin{enumerate}
        \item Imposibilidad de que una letra sea codificada como ella misma
        \item Se trata de una sustitución simple aplicada a los pares de letras en la que existe una correspondencia unívoca entre cada par de letras y su cifra.
        \item Cada letra puede sustituirse, exclusivamente, por las que comparten con ella línea o columna en el cuadro, lo que no hace más que ocho en total y puede revelar la estructura del cuadro.
        \item Dos pares de letras que sean invertidos darán lugar a dos nuevos pares de letras también invertidos
    \end{enumerate}


    Nota: El código asociado al algoritmo puede encontrarse \href{https://github.com/Gaspect/playfair}{aquí}. 
\end{document}